\documentclass[12pt]{article}
\usepackage[margin=1.0in]{geometry}
\usepackage{graphicx}
\usepackage[font=footnotesize, skip=2pt]{subcaption}
\usepackage{float}
\usepackage{datetime}
\usepackage{amsmath}
\usepackage[font=small, skip=3pt]{caption}
\usepackage[%  
    colorlinks=true,
    pdfborder={0 0 0},
    linkcolor=red
]{hyperref}


\title{CTA200H Project}
\author{Lisa Nasu-Yu \small{lisa.nasu.yu@mail.utoronto.com}\\
Prof. Norm Murray, Fergus Horrobin\\
Supervisor: Prof. Abigail Crites, Dr. Dongwoo Chung}
\date{\today}

\begin{document}

\maketitle
In this project, we transform Tomographic Ionized Carbon Intensity Mapping Experiment (TIME) data into maps through intensity mapping of ionized carbon (C$_{II}$). The region in the sky scanned in our set of data, described by right ascension and declination, is shown in Figure \ref{fig:coord}. We can see some jumps in the telescope's movement, resulting in a map that was not exactly quadrilateral. A polynomial fit of degree 2 was removed from the time ordered data and then mapped. The data was taken with a variety of detectors; an example timestream and map are shown in Figure \ref{fig:map}.

\begin{figure}[H]
	\centering
	\begin{subfigure}{0.5\linewidth}
		\centering
		\includegraphics[width=\linewidth]{ra.pdf}
		\subcaption{}
	\end{subfigure}%
	\begin{subfigure}{0.5\linewidth}
			\centering
			\includegraphics[width=\linewidth]{dec.pdf}
			\subcaption{}
	\end{subfigure}
	\caption{Right ascension (left) and declination (right) throughout timestream.}
	\label{fig:coord}
\end{figure}

\begin{figure}[H]
	\centering
	\begin{subfigure}{0.5\linewidth}
		\centering
		\includegraphics[width=\linewidth]{timestream.pdf}
		\subcaption{}
	\end{subfigure}
	\begin{subfigure}{1.3\linewidth}
			\centering
			\includegraphics[width=\linewidth]{map.pdf}
			\subcaption{}
	\end{subfigure}
	\caption{Timestream (top) and map (bottom) from feed horn 3, channel 6.}
	\label{fig:map}
\end{figure}

\end{document}